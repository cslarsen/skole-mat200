\documentclass[a4paper,norsk,12pt]{article}
\usepackage[utf8]{inputenc}

% Oppsett for norsk
\usepackage[norsk]{babel}
\usepackage{times}
\usepackage[T1]{fontenc}
\usepackage{parskip}
\DeclareUnicodeCharacter{00A0}{ }
\newcommand{\strek}{\textthreequartersemdash}

% Andre pakker
\usepackage{oving}
\usepackage{amsmath}
\usepackage{amssymb}
\usepackage{varioref}
\usepackage{subcaption}
\usepackage{units}
\usepackage{todo}

\def \oppgavename {Problem}

% Roman numerals
\makeatletter
\newcommand*{\rom}[1]{\expandafter\@slowromancap\romannumeral #1@}
\makeatother


\title{MAT200 --- Mathematical Methods 2}
\subtitle{Compulsory Assignment 3}
\author{Christian Stigen}
\date{UiS, 15.~april, 2016}

\begin{document}
\maketitle

\oppgave{1 (i)}
By Gauss-Jordan elimination,
\begin{align*}
    \left[
      \begin{array}{rrr|r}
        1 & 1 & 1 & 2 \\
        2 & 3 & 4 & 4 \\
        4 & 3 & 1 & 9
      \end{array}
    \right]
    & \rightarrow
    \left[
      \begin{array}{rrr|r}
         1 &  1 &  1 &  2 \\
         0 &  1 &  2 &  0 \\
         0 & -1 & -3 &  1 \\
      \end{array}
    \right]
    \rightarrow
    \left[
      \begin{array}{rrr|r}
         1 &  0 & -1 &  2 \\
         0 &  1 &  2 &  0 \\
         0 &  0 & -1 &  1 \\
      \end{array}
    \right]
    \\
    & \rightarrow
    \left[
      \begin{array}{rrr|r}
         1 &  0 & -1 &  2 \\
         0 &  1 &  2 &  0 \\
         0 &  0 &  1 & -1 \\
      \end{array}
    \right]
    \rightarrow
    \left[
      \begin{array}{rrr|r}
         1 &  0 &  0 &  1 \\
         0 &  1 &  0 &  2 \\
         0 &  0 &  1 & -1 \\
      \end{array}
    \right]
\end{align*}
giving the solution $x=1$, $y=2$ and $z=-1$.

\oppgave{1 (ii)}
\begin{align*}
    \left[
      \begin{array}{rrr|r}
        1 &  1 &  1 &  3 \\
        1 &  3 &  5 &  9 \\
        3 &  1 & -1 &  3
      \end{array}
    \right]
    \rightarrow
    \left[
      \begin{array}{rrr|r}
        1 &  1 &  1 &  3 \\
        0 &  2 &  4 &  6 \\
        0 & -2 & -4 & -6
      \end{array}
    \right]
    \rightarrow
    \left[
      \begin{array}{rrr|r}
        1 &  0 & -1 &  0 \\
        0 &  1 &  2 &  3 \\
        0 &  0 &  0 &  0
      \end{array}
    \right]
\end{align*}
We can't go further, but we see from the first row that $x=z$, and by the
second that $y+2z=3$ or $y=3-2z$. Thus either $z$ or $x$ can be freely chosen.

\oppgave{1 (iii)}
\begin{align*}
    \left[
      \begin{array}{rrrr|r}
        1 & -1 &  1 & -1 &  1 \\
        1 &  1 & -1 & -1 &  2 \\
        1 &  0 &  0 & -1 &  4
      \end{array}
    \right]
    & \rightarrow
    \left[
      \begin{array}{rrrr|r}
        1 & -1 &  1 & -1 &  1 \\
        0 &  2 & -2 &  0 &  1 \\
        0 &  1 & -1 &  0 &  3
      \end{array}
    \right]
    & \rightarrow
    \left[
      \begin{array}{rrrr|r}
        1 & -1 &  1 & -1 &  1 \\
        0 &  1 & -1 &  0 &  3 \\
        0 &  2 & -2 &  0 &  1
      \end{array}
    \right]
    \\
    & \rightarrow
    \left[
      \begin{array}{rrrr|r}
        1 &  0 &  0 & -1 &  4 \\
        0 &  1 & -1 &  0 &  3 \\
        0 &  0 &  0 &  0 & -5
      \end{array}
    \right]
\end{align*}
The last row gives $0=-5$, meaning \textit{the system is inconsistent}.

\oppgave{2}
\begin{align*}
  A &=
  \left[
    \begin{array}{rrr}
      1 &  5 &  0 \\
      0 & -6 &  a \\
      1 &  a &  1
    \end{array}
  \right]
\end{align*}
We'll start by looking at $\det{(A)}= 0$,
\begin{align*}
  \det{(A)} = (-6-a^2) +5a &= -a^2+5a-6 \\
  -a^2+5a-6 &= 0 \\
   a^2-5a+6 &= 0 \\
  a & \in \{2, 3\}
\end{align*}
The system has unique solutions for $a \notin \{2, 3\}$. But we're after infinite
solutions, so let's look at $a=2$.
\begin{align*}
      x + 5y &= 6 \\
    -6y + 2z &= p \\
  x + 2y + z &= p + 4
\end{align*}
Using Gauss-Jordan,
\begin{align*}
  \left[
    \begin{array}{rrr|r}
      1 &  5 &  0 & 6 \\
      0 & -6 &  2 & p \\
      1 &  2 &  1 & p+4
    \end{array}
  \right]
  & \rightarrow
  \left[
    \begin{array}{rrr|r}
      1 &  5 &  0 & 6 \\
      1 &  2 &  1 & p+4 \\
      0 & -6 &  2 & p
    \end{array}
  \right]
  & \rightarrow
  \left[
    \begin{array}{rrr|r}
      1 &  5 &  0 & 6 \\
      0 & -3 &  1 & p-2 \\
      0 & -6 &  2 & p
    \end{array}
  \right]
  \\
  \rightarrow
  \left[
    \begin{array}{rrr|r}
      1 &  0 &  5/3 & 6+\frac{5(p-2)}{3} \\
      0 & -3 &  1 & p-2 \\
      0 &  0 &  4 & 3p-4
    \end{array}
  \right]
\end{align*}
\textit{Not completed.}

\oppgave{3 (i)}
\begin{align*}
  \left[
    \begin{array}{ccc}
      1 & 0 & 0 \\
      1 & 1 & 1 \\
      0 & 2 & 1
    \end{array}
  \right]
  \left[
    \begin{array}{ccc}
      1 & 3 \\
      2 & 4 \\
      1 & 5
    \end{array}
  \right]
  &=
  \left[
    \begin{array}{ccc}
      1 & 3 \\
      4 & 12 \\
      5 & 13
    \end{array}
  \right]
\end{align*}

\oppgave{3 (ii)}
\begin{align*}
  \left[
    \begin{array}{ccc}
      1 &  2 \\
      1 &  3 \\
      3 & -1
    \end{array}
  \right]
  \left[
    \begin{array}{ccc}
      1 & 0 \\
     -1 & 2
    \end{array}
  \right]
  &=
  \left[
    \begin{array}{ccc}
     -1 & 4 \\
     -2 & 6 \\
      4 & -2
    \end{array}
  \right]
\end{align*}

\oppgave{3 (iii)}
\begin{align*}
  \left[
    \begin{array}{ccc}
     -1 &  1 \\
      3 & -1
    \end{array}
  \right]
  \left[
    \begin{array}{ccc}
      1 & 2 & 3 \\
      4 & 5 & 6
    \end{array}
  \right]
  &=
  \left[
    \begin{array}{ccc}
      3 & 3 & 3 \\
      -1 & 1 & 3
    \end{array}
  \right]
\end{align*}

\oppgave{4 (i)}
\begin{align*}
  \left|
    \begin{array}{ccc}
      1 & 0 & 0 \\
      1 & 2 & 0 \\
      1 & 2 & 3 \\
    \end{array}
  \right|
  &= 1\cdot(6-0) - 0 + 0 = 6
\end{align*}

\oppgave{4 (ii)}
\begin{align*}
  \left|
    \begin{array}{ccc}
      1 & 2 & 3 \\
      2 & 1 & 3 \\
      1 & 4 & 5 \\
    \end{array}
  \right|
  &= 1\cdot-7 -2\cdot7 + 3\cdot-7
  = (-3+3)\cdot7=0
\end{align*}

\oppgave{4 (iii)}
\begin{align*}
  \left|
    \begin{array}{ccc}
      0 & 1 & 0 \\
      1 & 0 & 1 \\
      0 & 1 & 0 \\
    \end{array}
  \right|
  &= 0-1\cdot0-0=0
\end{align*}

\oppgave{4 (iv)}
\begin{align*}
  \left|
    \begin{array}{ccc}
      0 & 1 & 2 \\
      1 & 0 & 1 \\
      2 & 1 & 0 \\
    \end{array}
  \right|
  &= 0 -1\cdot-2 +2\cdot1 = 4
\end{align*}

\oppgave{4 (v)}
\begin{align*}
  \left|
    \begin{array}{cccc}
      1 & 3 & 6 & 1 \\
      0 & 4 & 5 & 0 \\
      0 & 5 & 4 & 0 \\
      2 & 6 & 3 & 1
    \end{array}
  \right|
  &=
  1\left|
    \begin{array}{ccc}
      4 & 5 & 0 \\
      5 & 4 & 0 \\
      6 & 3 & 1
    \end{array}
  \right|
  + 0 + 0 - 2
  \left|
    \begin{array}{ccc}
      3 & 6 & 1 \\
      4 & 5 & 0 \\
      5 & 4 & 0
    \end{array}
  \right|
  \\
  &=
  (4\cdot4 -5\cdot5) - 2\cdot(3\cdot5-6\cdot0+1(16-25))
  = 9
\end{align*}

\oppgave{5 (i)}
We construct a matrix $A$ and know that if $\det{(A)} \ne 0$, the given vectors
are linearly independent.
\begin{align*}
  A &=
  \left[
    \begin{array}{ccc}
      0 & 1 & 1 \\
      1 & 0 & 2 \\
      0 & 1 & 3
    \end{array}
  \right] \\
  \det{(A)} &= 0 -3 + 1 = -2
\end{align*}
As the determinant $\ne 0$, the vectors are linearly independent.

\oppgave{5 (ii)}
\begin{align*}
  \left|
    \begin{array}{ccc}
      1 & 3 & 2 \\
      2 & 2 & 2 \\
      3 & 1 & 2
    \end{array}
  \right|
  &= 2+6-8 = 0
\end{align*}
The vectors are linearly independent.

\oppgave{5 (iii)}
\begin{align*}
  \left|
    \begin{array}{ccc}
      1 & 2 \\
      3 & 7
    \end{array}
  \right|
  &= 7-6 = 1
\end{align*}
Thus, $u_1$ and $u_2$ are linearly independent.
\begin{align*}
  \left|
    \begin{array}{ccc}
      2 & 15 \\
      7 & 9
    \end{array}
  \right|
  &= 18-7\cdot15 \ne 0
\end{align*}
Thus, $u_2$ and $u_3$ are also linearly independent. That means all vectors are
linearly independent, by transitivity.

\oppgave{5 (iv)}
\textit{Not completed.}

\end{document}
