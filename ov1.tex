\documentclass[a4paper,norsk,12pt]{article}
\usepackage[utf8]{inputenc}

% Oppsett for norsk
\usepackage[norsk]{babel}
\usepackage{times}
\usepackage[T1]{fontenc}
\usepackage{parskip}
\DeclareUnicodeCharacter{00A0}{ }
\newcommand{\strek}{\textthreequartersemdash}

% Andre pakker
\usepackage{oving}
\usepackage{amsmath}
\usepackage{amssymb}
\usepackage{varioref}
\usepackage{subcaption}
\usepackage{units}
\usepackage{todo}

\def \oppgavename {Problem}

% Roman numerals
\makeatletter
\newcommand*{\rom}[1]{\expandafter\@slowromancap\romannumeral #1@}
\makeatother


\title{MAT200 --- Mathematical Methods 2}
\subtitle{Compulsory Assignment 1}
\author{Christian Stigen}
\date{UiS, 19.~februar, 2016}

\begin{document}
\maketitle

\oppgave{1 (i)}
\label{problem.1}

The radicand must be zero or greater, or
\begin{align*}
  0 & \leqslant x^2-6x+\frac{1}{4}y^2 \> \Big{|} \cdot-4 \\
  0 & \geqslant 4x(6-x)-y^2 \\
  y^2 & \geqslant 4x(6-x)
\end{align*}
The right-hand side is zero for $x=0$ and $x=6$, and negative for $x<0$
and $x>6$. Thus, the inequality is true for $0 < x < 6$.

For plotting $\mathcal{D}(f)$, we note that
  \begin{align}
    y_{-} \leqslant -2\sqrt{x(6-x)} & \wedge y_{+} \geqslant 2\sqrt{x(6-x)}
    \,\text{for}\, 0<x<6 \label{eq.1i}
  \end{align}
In fact, taken together, they form an ellipse (see figure \vref{plot.p1}).

\oppgave{1 (ii)}

$C=0$ is given by (\ref{eq.1i}) above,
\begin{align*}
  f(x,y) = \sqrt{x^2 - 6x + \frac{1}{4}y^2} &= 0\\
  x^2 - 6x + \frac{1}{4}y^2 &= 0\\
  y = \pm2\sqrt{x(6-x)} \, \text{for} \, &0 \leqslant x \leqslant 6
\end{align*}
$C=4$ is given by
\begin{align*}
  f(x,y) &= \sqrt{x^2 - 6x + \frac{1}{4}y^2} = 4\\
  \frac{1}{4}y^2 &= 16 - x^2 + 6x\\
  y &= \pm2\sqrt{(x+2)(8-x)}
\end{align*}

The plot of these two level-curves (or \textit{contour curves}) is given in
figure \vref{plot.p2}. Note that we could have used scaling and shifting, which
is much easier, but I felt like doing it this way for fun.

\oppgave{1 (iii)}
Any level-curve is defined by $f(x,y)=C$. For the one corresponding to $(5,6)$,
we could insert $x_0=5$ and $y_0=6$ and find $C$, but we don't need to: When we
perform implicit differentiation of the resulting expression---to find the
slope in $(5,6)$, or any point, in fact---we see that $C'=0$. Therefore we go
straight to differentiation.
\begin{align*}
  C &= x^2 - 6x + \frac{1}{4}y^2 \\
  0 &= 2x - 6 + \frac{1}{2}yy' \\
  y' &= \frac{12-4x}{y} \\
  y'(5,6) &= -\frac{4}{3}
\end{align*}
Now that we have the slope, we can find the tangent line going through this
point by back-calculating the $y$-value for $x=0$ to find $\ell$.
\begin{align*}
  \ell_{(x_0, y_0)} &= y_0 - x_0y'(x_0,y_0) + y'(x_0,y_0)x \\
  \ell_{(5,6)} &= 6+\frac{4\cdot5}{3} -\frac{4}{3}x \\
  \ell_{(5,6)} &= -\frac{4}{3}x -\frac{38}{3}
\end{align*}
This is all plotted in figure \vref{plot.p3}.

\begin{figure}[h]
  \centering
  \includegraphics{ob1plot.png}
  \caption{Plot of $\mathcal{D}(f)$ in problem 1 (i).}
  \label{plot.p1}
\end{figure}

\begin{figure}[h]
  \centering
  \includegraphics{ob1plot2.png}
  \caption{Plot of $C=0$ and $C=4$ in problem 1 (ii).}
  \label{plot.p2}
\end{figure}

\begin{figure}[h]
  \centering
  \includegraphics{ob1plot3.png}
  \caption{Plot of tangent line $\ell_{(5,6)}$ in problem 1 (iii).}
  \label{plot.p3}
\end{figure}

\oppgave{2 (i)}

\oppgave{2 (ii)}

\oppgave{2 (iii)}

\oppgave{3 (i)}

\oppgave{3 (ii)}

\oppgave{3 (iii)}

\oppgave{3 (iv)}

\begin{align*}
  F(0,y) &= 1 \\
  2y \sin^2(-\frac{\pi}{4}) &= 1 \\
  2y\frac{1}{2} &= 1 \\
  y &= 1 \\
  F(0,1) &= 1
\end{align*}

\end{document}
